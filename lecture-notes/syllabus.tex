\documentclass[a4paper,12pt]{article}

%\usepackage{titling}
%\setlength{\droptitle}{-4cm}
%\predate{}
%\postdate{}
%\preauthor{}
%\postauthor{}

\usepackage{amsmath,amssymb,amsthm,mathtools}
\usepackage{bm}
\usepackage{booktabs,hyperref}
\usepackage[longnamesfirst]{natbib}
\bibliographystyle{ecta}

\begin{document}
\noindent {\LARGE Topics in Distributional Macroeconomics}

\medskip \noindent {\large Elective Course, Tinbergen Institute, May/June 2022}

\subsection*{Lecturers}

Fabian Greimel (coordinator), \url{https://www.greimel.eu}

\noindent
Stefanie J. Huber, \url{https://sites.google.com/site/stefaniehuber/}

\noindent
Enrico Perotti, \url{https://www.enricoperotti.eu}


\subsection*{Description}

The term ``distributional macroeconomics'' was introduced by Benjamin Moll as a replacement for ``macroeconomics with heterogeneous agents'' to promote the view that the \emph{macroeconomy is a distribution} of state variables (e.g.\ income and wealth).\footnote{See \url{ https://benjaminmoll.com/wp-content/uploads/2019/07/DM_long.pdf}.}

This course will explore some implications of income and wealth heterogeneity for macroeconomic dynamics and macroeconomic policy. First, we want to understand how secular trends (falling interest rates and rising debt) are connected to rising inequality. And second, we want to understand how heterogeneity matters for the aggregate response to macroeconomic shocks.

The focus of the course is on topics, rather than computational methods. (These are covered in second year electives.) We will use simplified models to replicate results from papers and compare the results to empirical evidence. For most topics, we will provide ready-to-run code in the form of interactive \href{https://github.com/fonsp/Pluto.jl}{\texttt{Pluto.jl}} notebooks. 

%The course is based on recent research papers. Most papers build on the Bewley-Huggett-Aiyagari model, some build on the two-agent (Saver-Spender) model. 

\subsection*{Topics}

\begin{enumerate}
\item Consumer Credit and Default (Greimel)
\item Household Balance Sheets and Recessions (Greimel)
\item Behavioral Macroeconomics (Huber)
\item Inequality, Interest Rates and Household Debt (Greimel \& Perotti)
\item Monetary Policy, Heterogeneity and HANK (Greimel)
\end{enumerate}

\subsection*{Evaluation}

\begin{enumerate}
\item Group homeworks
\item Presentation of a paper
\item Research proposal
\end{enumerate}

\subsection*{Structure and Content}

\paragraph{Week 1: Introduction and consumer credit (Greimel)}
\begin{enumerate}
\item Introduction
  \begin{enumerate}
  \item Heterogeneity in Macroeconomics: Some History (Based on lecture notes by Ben Moll)
  \item Policy relevance
  \item Modeling income and wealth inequality
  \item Modeling secured and unsecured debt
  \item infinite lifetime vs perpetual youth vs lifecycle model
  \end{enumerate}

\item Consumer Credit and Default \citep{athreya2002welfare,chatterjee2007quantitative,livshits2007consumer}, surveyed by \citet{exler2020consumer}
\end{enumerate}

\paragraph{Week 2: Household balance sheets (Greimel)}

\begin{enumerate}
\item Household Balance Sheets and the Great Recession
  \begin{enumerate}
  \item \cite{mian2013household}
  \item \cite{berger2015consumption}
  \end{enumerate}
\item Housing Wealth Effects: How Consumption Reacts to a House Price Bust
  \begin{enumerate}
  \item \cite{berger2018house}
  \item \cite{guren2021housing}
  \item \cite{greimel2019understanding}
  \end{enumerate}
\end{enumerate}

\paragraph{Week 3: Behavioral macroeconomics (Huber)}
\begin{enumerate}
\item Empirical Evidence on cross-country heterogeneity in House Price Booms and Busts \citep{Huber:2019}
\item The Rent versus Buy Decision: User Cost Approach
  versus Behavioral Concepts
  \citep{Huber:2022,Huber:2020}
\item Social Networks, Experience Effects and Expectations 
  \citep{Kuchler:2019, bailey2018housing, Malmendier:2019, kaplan2020housing}
\end{enumerate}


\paragraph{Week 4: Inequality, interest rates and household debt in the long-run}

\begin{enumerate}
  
\item First Part (Greimel)
  \begin{itemize}
  \item Saving Glut of the Rich  \citep{kumhof2015inequality,mian2021indebted-demand}
  \item Social comparisons and trickle-down consumption
    \citep{bertrand2016trickle, bellet2019mcmansion, drechsel2021falling-behind}
  \end{itemize}
  
\item Second Part (Perotti)
  \begin{itemize}
  \item Redistributive Growth \citep{doettling2020secular}
  \end{itemize}
\end{enumerate}


\paragraph{Week 5: Monetary Policy, Heterogeneity and HANK (Greimel)}
\begin{itemize}
\item Microeconomics Heterogeneity and Macroeconomic Shocks \citep{kaplan2018microeconomic}
\item Original HANK \citep{kaplan2018monetary}
\item Simple HANK \citep{bilbiie2018analytical}
\end{itemize}

\paragraph{Weeks 6 \& 7: Student presentations}

\begin{itemize}
\item Behavioural Consumer Credit
  \begin{itemize}
  \item over-optimism: \citep{exler2020over-optimistic}
  \item present bias \citep{laibson2021present}
  \end{itemize}
\item Other
  \begin{itemize}
  \item \citet{bailey2019beliefs-leverage}
  \item \citet{laibson2022mpc-to-mpx}
  \item more suggestions to follow
  \item \dots or make your own suggestion
  \end{itemize}
\end{itemize}

% \section{Models \& Code}

% \begin{enumerate}
% \item Consumer debt and default [Q]
% \item Simple Model of Housing [A]
% \item Housing and Renting [Q] (cf slides)
% \item Discrete-time vs continuous time (infinitesimal operators bla bla)
% \end{enumerate}


\bibliography{../inequality.bib}
\end{document}
