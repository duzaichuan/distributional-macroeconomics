\documentclass[aspectratio=169,mathserif]{beamer}

\useoutertheme[numbering=fraction]{metropolis}
\useinnertheme{metropolis}
\usefonttheme{metropolis}
\usecolortheme{default}

\usepackage{array}
\usepackage{subcaption}
\usepackage{graphicx}
\usepackage[longnamesfirst]{natbib}
\usepackage{minibox}
\usepackage{amsmath,centernot}
% \input{../header-files/src/metropolis-titlepage}
% \input{../header-files/src/header.tex}
% \input{../header-files/src/acronyms.tex}
% \input{../header-files/src/math-def.tex}
% \input{../header-files/src/tikz-def.tex}
% %\input{../header-files/src/beamer-header.tex}
% \input{../header-files/src/falling-behind.tex}

\usepackage{pgfplots}
\usetikzlibrary{tikzmark, positioning}

\AtBeginSection[]
{
  \frame{\sectionpage}
  % \begin{frame}
  %   \frametitle{Outline}
  %   \tableofcontents[currentsubsection]
  % \end{frame}
}

\newcommand\blfootnote[1]{%
  \begingroup
  \renewcommand\thefootnote{}\footnote{#1}%
  \addtocounter{footnote}{-1}%
  \endgroup
}

\newcommand{\credittoben}{\blfootnote{Taken from Ben Moll's slides with only minor adjustments}}

 \author{Fabian Greimel}
\title{Topics in Distributional Macroeconomics}

\begin{document}
 
\frame[plain]{\maketitle}

\begin{frame}{What is \emph{Distributional Macroeconomics}?}

  \begin{itemize}
  \item basic point of view: the \alert{macroeconomy is a distribution} (of productivities, incomes, wealth)
  \item Goals
    \begin{enumerate}
    \item study of \alert{macroeconomic questions in terms of distributions} rather than just aggregates
      
    \item study \alert{rich intereactions} of heterogeneity and macroeconomic aggregates
    \end{enumerate}
  \item Methods
    \begin{enumerate}
    \item \alert{models with heteregeneous agents} (agents or firms)
    \item use \alert{micro data} (e.g. household surveys) and \alert{macro data} (e.g. time series and country panels) to inform models
    \end{enumerate}
  \end{itemize}
\end{frame}

\begin{frame}{Questions we address in this course}
  \begin{enumerate}
  \item Why have household debt and bankruptcies risen prior to 2007?
  \item How do bankruptcy laws affect social welfare?
  \item (How) has the housing boom prior to 2007 amplified the Great Recession?
  \item Where did the housing boom come from?
  \item ... And what role does inequality play in all theses questions?
  \end{enumerate}
\end{frame}

\begin{frame}{Course logistics: Everything is in flux}
  \begin{columns}
    \begin{column}{0.5\linewidth}
      \begin{block}{Background}
        \begin{itemize}
        \item designed from scratch
        \item \emph{the course that I've always wanted}
        \item \dots{} but things take time
        \item your feedback will shape this course     
        \end{itemize}
      \end{block}
      \begin{block}{Your Tasks}
        \begin{enumerate}
        \item interact with me
        \item graded assignments
        \item present a paper
        \item write a short research proposal
        \end{enumerate}
      \end{block} 
    \end{column}
    \begin{column}{0.5\linewidth}
      \begin{block}{Assignments}
        \begin{itemize}
        \item focus on topics, not methods
        \item start from runnable code
        \end{itemize}
      \end{block}
      \begin{block}{Presentation}
        \begin{itemize}
        \item paper \emph{suggestions} in the syllabus
        \item starting point for your proposal?
        \end{itemize}
      \end{block}
      \begin{block}{Goals of Research Proposal}
        \begin{itemize}
        \item focus your interests
        \item generate original idea(s)
        \item get you started with your research
        \end{itemize} 
      \end{block}
    \end{column}
  \end{columns}


\end{frame}



\begin{frame}{Inequality in macroeconomics: A brief history of economic hought\credittoben}
  \begin{itemize}
  \item before modern macro: 1930 to 1970
  \item 1st generation modern macro: 1970 to 1990
  \item 2nd generation modern macro: 1990 to financial crisis
  \item 3rd generation modern macro: after the financial crisis
  \end{itemize}
  
  Main drivers of evolution in modern macro era
  \begin{enumerate}
  \item better data
  \item better computers \& algorithms
  \item current events (rising inequality, financial crisis)
  \end{enumerate}

  
\end{frame}

\begin{frame}{Before modern macro: 1930 to 1970\credittoben}
  \begin{itemize}
  \item Keynesian IS/LM: about aggregates, no role for inequality/distribution by design
  \item Distribution does play role in growth theory 
    \begin{itemize}
    \item mostly \alert{factor} income distribution (capital vs labor): Kaldor, Pasinetti and other Cambridge UK theorists
    \item rarely \alert{personal} income distribution: e.g. Stiglitz, Blinder 
    \end{itemize}
  \item Disconnected empirical work on inequality (Kuznets)
  \end{itemize}
\end{frame}

\begin{frame}{First Generation Macro Theories: 1970 to 1990\credittoben}
  \begin{itemize}
  \item representative agent (RA) models, e.g.\ RBC \& New Keynesian (NK) models
  \item about aggregates, no role for inequality/distribution by design
  \item advertised as ``microfounded'' but representative agent assumption cuts it from much of micro research
    
  \item What's wrong with that?
    \begin{enumerate}
    \item cannot speak to a number of important empirical facts, e.g.
      \begin{itemize}
      \item unequally distributed growth
      \item poorest hit hardest in recessions
      \end{itemize}
    \item cannot think coherently about welfare --- ``who gains, who loses?''
    \end{enumerate}
  \end{itemize}
\end{frame}

\begin{frame}{Second Generation Macro Theories: 1990 to 2008\credittoben}
  \begin{itemize}
  \item incorporate income and wealth heterogeneity from micro data
  \item represent economy with a distribution \dots
  \item \dots{} that moves over time, responding to macroeconomic shocks and policies
  \item referred to as \alert{\emph{heterogeneous agent (HA) models}}
  \item important early contributions by Aiyagari, Bewley, Huggett, Krusell-Smith, Den Haan, \dots
  \item can potentially speak to
    \begin{itemize}
    \item unequally distributed growth
    \item poorest hit hardest in recessions
    
    \end{itemize}
  \item and are useful for welfare analysis
  \end{itemize}
\end{frame}

\begin{frame}{Second Generation (continued): Inequality $\centernot\implies$ Macro\credittoben}
  \begin{itemize}
  \item heterogeneity doesn't matter much for aggregates (\alert{``inequality  $\centernot\implies$ macro'' })
  \item Reason: in these theories, \alert{rich $=$ scaled version of poor}
    \begin{itemize}
    \item rich and poor differ in wealth \dots
    \item \dots{} but not consumption and saving behavior
    \end{itemize}
  \item Problem: in data, \alert{rich $\ne$ scaled version of poor}, e.g. rich have
    \begin{itemize}
    \item lower MPCs out of transitory income changes
    \item higher saving rates out of permanent income, wealth
    \end{itemize}
  \item Note: some important contributions from this period don’t fit this narrative
    \begin{itemize}
    \item Banerjee-Newman, Benabou, Galor-Zeira, Persson-Tabellini, \dots
    \item also related: 1950s “capitalist-worker theories” of Kaldor, Pasinetti, \dots
    \end{itemize}
  \end{itemize}
\end{frame}

\begin{frame}{Third Generation Theories: after the Crisis\credittoben}
  \begin{itemize}
  \item \alert{take micro data more seriously}
  \item emphasize things like
    \begin{itemize}
    \item household balance sheets
    \item credit constraints
    \item MPCs that are high on average but heterogeneous
    \item non-homotheticities, non-convexities
    \end{itemize}
  \item Typical finding: distribution matters for macro (\alert{``inequality  $\implies$ macro'' })
  \item Will see a number of examples throughout the course
\end{itemize}
\end{frame}

\begin{frame}{This course is about ``3rd generation'' models\credittoben}
  \begin{itemize}
    %\item Methods for solving them and some fun applications
  \item \emph{Distributional macroeconomics} is hard
      \begin{itemize}
      \item closed-form solutions are rare
      \item computations are challenging
      \item large micro datasets that may be hard to think through
      \item (Note: even though models harder to solve, they are often easier to understand --- you have good intuition about micro behavior!)
      \end{itemize}
    \item Why should you be interested in this?
      \begin{itemize}
      \item fertile area of research, excellent dissertation topics!
      \item many open questions
      \item economics is becoming more empirical, macro no exception
      % \item pays off to be a bit strategic in your choice of topic
      \end{itemize}
  \end{itemize}
 
\end{frame}

\section{Literature}
%\input{literature}

\section{Model}
%\input{model}

\section{Results}
%\input{results}

\section{Summary}
%\input{conclusion}


\appendix
% \backupbegin

\begin{frame}[allowframebreaks]{Literature}
  \bibliographystyle{ecta} 
  \bibliography{../inequality}
\end{frame}


\section{Back-up slides}

\begin{frame}{Back-up 1}
  bla
\end{frame}
\begin{frame}{Back-up 2}
  bla bla   
\end{frame}


% \backupend


\end{document}

%%% Local Variables:
%%% TeX-engine: luatex
%%% TeX-master: t
%%% End:
